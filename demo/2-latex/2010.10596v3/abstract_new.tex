%Machine learning is increasingly taking hold of life-impacting decisions. Therefore, it is essential to hold machine learning models accountable, and explainable to the people it is affecting. \jpd{We don't need either of these sentences.}
Machine learning plays a role in many deployed decision systems, often in ways that are difficult or impossible to understand by human stakeholders.  Explaining, in a human-understandable way, the relationship between the input and output of machine learning models is essential to the development of trustworthy machine learning based systems.  A burgeoning body of research seeks to define the goals and methods of \emph{explainability} in machine learning.  In this paper, we seek to review and categorize research on \emph{counterfactual explanations}, a specific class of explanation that provides a link between what could have happened had input to a model been changed in a particular way.  Modern approaches to counterfactual explainability in machine learning draw connections to the established legal doctrine in many countries, making them appealing to fielded systems in high-impact areas such as finance and healthcare.  Thus, we design a rubric with desirable properties of counterfactual explanation algorithms and comprehensively evaluate all currently proposed algorithms against that rubric.  Our rubric provides easy comparison and comprehension of the advantages and disadvantages of different approaches and serves as an introduction to major research themes in this field.  We also identify gaps and discuss promising research directions in the space of counterfactual explainability. 
